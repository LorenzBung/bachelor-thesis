% !TEX root = ../thesis.tex
\documentclass[11pt,a4paper,oneside]{report}		                     

% Mathe
\usepackage{marvosym}		% Package für Euro&Co.
\usepackage{amsfonts}		% Package für Mengensymbole (IN, IR, ...)
\usepackage{amsmath}		% Package für restlichen Mathekram
\usepackage{amssymb}        % math. Symbole und Umgebungen
\usepackage{mathtools}
\usepackage{unicode-math}   % Erlaubt die Mathe Schrift zu setzen

% Use german umlaute
%\usepackage{ngerman}
%\usepackage[T1]{fontenc}
%\usepackage[utf8]{inputenc}
\usepackage[english]{babel}

% Zitierung und Literaturverzeichnis
\usepackage{cite}
%\usepackage{apacite}		% Zitierstil
%\usepackage{natbib} 		% Erweiterte zitiermöglichkeiten 
\usepackage{babelbib}		% Übersetzung des Literaturverzeichnises und der Zitate

% Fonts
\usepackage{fontspec}
\setmainfont[Path=fonts/, BoldFont=swis721-Bold.ttf, ItalicFont=swis721-Italic.ttf, BoldItalicFont=swis721-Bold-Italic.ttf]{swis721.ttf}
\setmonofont[Path=fonts/, BoldFont=CourierNew-Bold.ttf, ItalicFont=CourierNew-Italic.ttf, BoldItalicFont=CourierNew-Bold-Italic.ttf]{CourierNew.ttf}
\setmathfont[Path=fonts/]{CambriaMath.ttf}


% formatting
\usepackage{a4}

% PDF als Cover Hintergrund
\usepackage[firstpage=true]{background}
\usepackage{geometry}
\usepackage{tikz}

% Verbesserte Darstellug der Schriftzeichen und Zwischenräumen
%\usepackage{lmodern}				% Benutze skalierbare Schriftfamilie
%\usepackage[activate]{microtype}

% Heading styles
\usepackage[sf,bf,raggedright]{titlesec}
\titlespacing*{\section}
{0pt}{5.5ex plus 1ex minus .2ex}{4.3ex plus .2ex}
\titlespacing*{\subsection}
{0pt}{5.5ex plus 1ex minus .2ex}{4.3ex plus .2ex}
%\titleformat{\chapter}[hang]{\Huge\bfseries}{\thechapter}{15pt}{\Huge\bfseries}

% Header and Footer
\usepackage{fancyhdr}

\fancypagestyle{fancybody}{
	\fancyhf{}
	\fancyhead[R]{\sc{\nouppercase{\leftmark}}}
	\fancyfoot[C]{\thepage}
}

\fancypagestyle{plain}{
	\fancyhf{}
	\fancyfoot[C]{\thepage}
	\renewcommand{\headrulewidth}{0pt}
}

\pagestyle{fancybody}

% Absatz einrücken unterdrücken
\setlength{\parindent}{0pt}

% Listings
\usepackage{listings}
\lstset{
	backgroundcolor=\color{lighter-gray},
	xleftmargin=\parindent,
	basicstyle=\ttfamily\footnotesize,
	breaklines=true,
	captionpos=b,
	showspaces=false,
	showstringspaces=false,
	frame=shadowbox,
	rulesepcolor=\color{ll-gray},
	rulecolor=\color{l-gray},
	belowskip=11pt,
}

% Depth of chapters and subsections
\setcounter{tocdepth}{3}
\setcounter{secnumdepth}{3}

% Url line break
\def\UrlBreaks{\do\/\do-}
\def\UrlBigBreaks{\do\:\do\/}

% Links
\usepackage[pdfstartview=FitV, 	% start in 'fit size height'-view
colorlinks=true, 	% Links farbig markieren
linkcolor=black, 	% Interne Links schwarz
citecolor=black, 	% Links zur Literatur schwarz
filecolor=black, 	% Links auf lokale Dateien schwarz
urlcolor=black,   	% Externe Links blau
breaklinks=true, 	% Links umbrechen
linktocpage=true 	% Im Inhaltsverzeichnis sind Seitenzahlen UND Text Links
]{hyperref}

%Move to the right.
\setlength{\hoffset}{5pt}

% Suppress warning of too small headheight
\headheight13.6pt

% Abkuerzungsverzeichnis
\usepackage[nohyperlinks, printonlyused]{acronym}
\usepackage[mode=buildnew]{standalone}

\usepackage[toc, page]{appendix}


%%%%%%%%%%%%
%% HTWG CI Colors
%%%%%%%%%%%%
\definecolor{htwg-white}{cmyk}{0.0,0.0,0.0,0.0}
\definecolor{htwg-black}{cmyk}{0.0,0.0,0.0,1.0}
\definecolor{htwg-black-90}{cmyk}{0.0,0.0,0.0,0.9}
\definecolor{htwg-black-80}{cmyk}{0.0,0.0,0.0,0.8}
\definecolor{htwg-black-70}{cmyk}{0.0,0.0,0.0,0.7}
\definecolor{htwg-black-60}{cmyk}{0.0,0.0,0.0,0.6}
\definecolor{htwg-black-50}{cmyk}{0.0,0.0,0.0,0.5}
\definecolor{htwg-black-40}{cmyk}{0.0,0.0,0.0,0.4}
\definecolor{htwg-black-30}{cmyk}{0.0,0.0,0.0,0.3}
\definecolor{htwg-black-20}{cmyk}{0.0,0.0,0.0,0.2}
\definecolor{htwg-black-10}{cmyk}{0.0,0.0,0.0,0.1}
\definecolor{htwg-soft-blue}{cmyk}{0.1,0.0,0.0,0.1}
\definecolor{htwg-dark-blue}{rgb}{0.165, 0.22, 0.29}
\definecolor{htwg-teal}{cmyk}{1.0,0.0,0.5,0.0}
