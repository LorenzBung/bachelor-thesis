% !TEX root = ../thesis.tex

\chapter{Zusammenfassung}
\label{ch:zusammenfassung}

Es wurde eine Pipeline zur Rekonstruktion von 3D-Polygonnetzen aus Punktwolken entwickelt und mehrere verschiedene Ansätze dafür getestet.
Die Segmentierungsalgorithmen \ac{RG}, \ac{ECE} und 2D-Watershed wurden implementiert.
Die Registrierung der Cluster wurde zunächst global mithilfe \ac{4PCS} geschätzt, um anschließend durch \ac{ICP} verfeinert zu werden.
Abschließend wurde die Punktwolke durch Poisson trianguliert und das resultierende Mesh nach Distanz von der Originalpunktwolke gefiltert.

Die Aufnahme eines Objekts aus verschiedenen Perspektiven mit einer fest montierten Kamera, anschließende Registrierung der Punktwolken und abschließende Poisson-Triangulation resultierte dabei im qualitativ hochwertigsten Mesh.

Bei der Segmentierung einer Aufnahme mehrerer sortenreiner Teile ist zudem aufgefallen, dass die gewählte Segmentierungsmethode stark variiert.
Einfluss haben sowohl Objekteigenschaften wie Größe, Form und Oberflächenstruktur, jedoch auch andere Faktoren wie die Distanz zwischen den Objekten.
Eine Entscheidung für einen Ansatz kann also nicht allgemein getroffen werden und ist unter Umständen auch nicht für ein Objekt eindeutig.

Die Qualität der Segmentierung hat sich stark in der Qualität des resultierenden Meshs wiedergespiegelt.
Bei Untersegmentierung stellten sich viele Cluster als unbrauchbar heraus und mussten verworfen werden.
Trat aber eine Übersegmentierung auf, war eine korrekte Registrierung jedoch in vielen Fällen nicht möglich.

%TODO