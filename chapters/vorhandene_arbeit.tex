% !TEX root = ../thesis.tex

\chapter{Vorhandene Arbeit}
\label{ch:vorhandene-arbeit}

In diesem Kapitel wird der aktuelle Stand der Forschung genannt sowie bereits existierende Lösungsansätze erklärt.


\section{Lokale Registrierung}
\label{sec:local-registration}

Bei der lokalen Registrierung von zwei Punktwolken müssen diese bereits grob aneinander ausgerichtet sein.
Der Registrierungsalgorithmus verfeinert diese Ausrichtung dann weiter.
Einer der bekanntesten Algorithmen zur lokalen Registrierung ist \ac{ICP} und seine vielen verschiedenen Varianten.


\subsection{\acl{ICP}}
\label{subsec:icp}

\subsection{Kinect Fusion}
\label{subsec:kinfu}

Durch die Veröffentlichung der Microsoft Kinect im Jahr 2010, einer 3D-Kamera für die Nutzung im Entertainment- und Gamingbereich, wurde durch den niedrigen Preis erstmals der breiten Masse der Zugang zu Tiefenkameras ermöglicht \cite[1:55]{kinfuTalkYoutube}.
\ac{KinFu} ist das Ergebnis einer Forschungsarbeit von Microsoft Research \cite{izadi2011kinectfusion} und bietet eine Möglichkeit, 3D-Rekonstruktionen in Echtzeit durchzuführen.

% Hier YAK erwähnen und erklären
YAK? \cite{klingensmith2015chisel}


\section{Globale Registrierung}
\label{sec:global-registration}

Bei der globalen Registrierung müssen sich die beiden Punktwolken nicht nahe der endgültigen Ausrichtung befinden - Translation und Rotation können beliebig sein.
Der Nachteil ist die häufig sehr hohe Komplexität der Algorithmen.
Außerdem bietet die globale Registrierung oft nur eine Grobregistrierung, weshalb es sich meist anbietet, anschließend noch eine lokale Registrierung zur Verbesserung der Ergebnisse durchzuführen.


\subsection{\acl{4PCS}}
\label{subsec:4pcs}

% Hier Super4PCS erwähnen und erklären