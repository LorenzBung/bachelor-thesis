% !TEX root = ../thesis.tex

\chapter{Einleitung}
\label{ch:einleitung}


\section{Motivation}
\label{sec:motivation}

Robotik im industriellen Kontext spielt in den letzten Jahren eine immer größer werdende Rolle für die Wirtschaft.
Insbesondere das autonome Greifen von unsortierten Teilen (sogenanntes Bin-Picking) durch einen Roboter kann viele Prozesse beschleunigen und somit wertvolle Zeit und Ressourcen sparen.

Zum Erkennen und Greifen der Objekte sind in vielen Fällen CAD-Modelle erforderlich \cite{drost2014recognition}.
Dies stellt häufig ein Problem dar, da ein Modell aus verschiedenen Gründen nicht vorhanden sein kann.
Etwa können die existierenden Modelle im aktuellen Fertigungszustand nicht vorhanden sein.
Ein weiterer Grund für fehlende Modelle ist, dass diese aus organisatorischen Gründen schwer zu bekommen sind, beispielsweise wenn nur eine Weiterverarbeitung eines zugelieferten Bauteils stattfindet.

Neben fehlenden Modellen sind auch häufig falsche Modelle ein Problem.
So kann es vorkommen, dass die existierenden Daten fertigungsbedingt nicht zum tatsächlichen Produkt passen und Teile daher nicht genau erkannt werden können.

Bei Bin-Picking-Anwendungen sind in vielen Fällen 3D-Kameras am Roboter verbaut.
Um schnell eine korrekte Darstellung der Objekte zu erhalten ist es eine Möglichkeit, die aufgenommenen 3D-Daten zur Generierung eines CAD-Modells zu verwenden.
Insbesondere bei wechselnden Produktkonfigurationen vereinfacht dies den Anwendungsprozess enorm.



\section{Zielsetzung}
\label{sec:zielsetzung}

In vielen Fällen liefern 3D-Kameras Daten in Form einer Punktwolke, eine Darstellung, auf die in \ref{subsec:punktwolken} genauer eingegangen wird.
Zur Rekonstruktion eines CAD-Modells aus einer solchen Punktwolke gibt es drei verschiedene Ansätze:

\begin{enumerate}
\item Eine am Roboterarm befestigte Kamera nimmt das Objekt aus verschiedenen Winkeln auf.
Die Kameraposition ist durch die Pose des Roboters gegeben.
\item Eine fest montierte, unbewegliche Kamera nimmt mehrere Aufnahmen eines Objekts auf, das zwischen den Aufnahmen bewegt wird.
Da die Position des Teils unbekannt ist, muss diese geschätzt werden.
\item Durch eine unbewegliche Kamera wird eine einzelne Aufnahme einer Kiste mit Objekten derselben Art aufgenommen.
Durch die unterschiedliche Orientierung der Objekte innerhalb der Kiste ergeben sich Daten aus mehreren Perspektiven.
\end{enumerate}

Ziel der Arbeit ist es, geeignete Lösungen für die verschiedenen Ansätze zu entwickeln sowie die unterschiedlichen Rekonstruktionsmöglichkeiten zu implementieren.

Weiterhin sollen die Ergebnisse untereinander und mit bereits bestehenden Algorithmen verglichen werden.
Dieser Vergleich soll sowohl bezüglich der wichtigsten Eigenschaft der Qualität, als auch auf Basis untergeordneter Faktoren wie Geschwindigkeit der Algorithmen und Nutzungskomfort der Ansätze stattfinden.



\section{Anwendungsbereich}
\label{sec:anwendungsbereich}

Entwurf, Implementierung und Evaluation der Ergebnisse finden mithilfe des proprietären Softwarepakets Mikado \cite{mikadoWebsite}, einem Projekt von isys vision GmbH \cite{isysWebsite}.
Zur Erfassung der Bilddaten kommen 3D-Kameras von Ensenso \cite{ensensoWebsite} zum Einsatz.
