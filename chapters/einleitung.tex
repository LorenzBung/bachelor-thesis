% !TEX root = ../thesis.tex

\chapter{Einleitung}
\label{ch:einleitung}


\section{Mikado}
\label{sec:mikado}

Isys vision GmbH \cite{isysWebsite} ist ein Unternehmen aus Freiburg im Breisgau, das sich mit Systemintegration und industrieller Bildverarbeitung beschäftigt.
Neben Entwicklungen in der 2D-Bildverarbeitung (zum Beispiel in der Leiterplattenproduktion) spielt auch Machine Vision im 3D-Bereich in Kombination mit Robotik eine große Rolle.


Mikado \cite{mikadoWebsite} ist ein Softwarepaket von isys vision, welches zur Robotersteuerung und 3D-Bildverarbeitung in der Industrie eingesetzt wird.
Es besteht aus zwei Komponenten: Mikado 3D dient der eigentlichen Bildverarbeitung, während Mikado Adaptive Robot Control (ARC) zusätzlich dazu die Robotersteuerung und Kollisionsplanung beinhaltet.
Die Hauptanwendung ist dabei das sogenannte ''Bin-Picking'', also das Greifen von sortenreinen Teilen aus einer unsortierten Kiste.
Das Softwarepaket basiert auf der Open-Source-Bibliothek \ac{ROS} \cite{quigley2009ros}.
Zur Erfassung der Bilddaten kommen 3D-Kameras von Ensenso \cite{ensensoWebsite} zum Einsatz.


\section{Motivation}
\label{sec:motivation}

Bei der Bestimmung der 6D-Posen von Objekten wird bei Mikado unter Anderem der ''surface based matching''-Algorithmus von MvTec Halcon \cite{drost2014recognition} verwendet.
Dafür wird ein CAD-Modell des Objekts benötigt, welches jedoch in vielen Fällen nicht vorhanden ist.

Die Gründe dafür sind vielfältig.
Etwa können die existierenden Modelle im aktuellen Fertigungszustand nicht vorhanden sein.
Ein weiterer Grund für fehlende Modelle ist, dass diese aus organisatorischen Gründen schwer zu bekommen sind, beispielsweise wenn nur eine Weiterverarbeitung eines zugelieferten Bauteils stattfindet.

Neben fehlenden Modellen sind auch häufig falsche Modelle ein Problem.
So kann es vorkommen, dass die existierenden Daten fertigungsbedingt nicht zum tatsächlichen Produkt passen und Teile daher nicht genau erkannt werden können.


Die Generierung eines CAD-Modells aus den Daten der 3D-Kamera ist somit eine Möglichkeit, um für das tatsächlich vorhandene Produkt eine korrekte Repräsentation zu finden, welche im weiteren Bildverarbeitungsprozess genutzt werden kann.
Weiterhin dient dies auch der Vereinfachung der Endanwendung von Mikado ARC.
Besonders bei oft variierenden Produktkonfigurationen wird der Anwendungsprozess vereinfacht, wenn nicht erst ein entsprechendes CAD-Modell organisiert werden muss.


\section{Zielsetzung}
\label{sec:ziel}

Für die Rekonstruktion aus der Punktwolke gibt es drei mögliche Ansätze:

\begin{enumerate}
\item Es liegt ein stationäres Objekt vor, welches mit einer am Roboter angebrachten 3D-Kamera erfasst wird.
Der Roboter bewegt die Kamera um das Objekt herum, um so Informationen aus mehreren Perspektiven zu erhalten.
Diese werden danach miteinander kombiniert; der Hintergrund der erfassten Daten wird anschließend eliminiert.

\item Die Kamera ist stationär angebracht, während das zu erkennende Objekt freihändig demonstriert wird.
So kann das Objekt aus mehreren Ansichten aufgenommen werden, welche dann zusammengeführt werden.

\item Es liegen mehrere Objekte der selben Art in verschiedenen Orientierungen vor, beispielsweise in einer Kiste.
Nach Segmentierung der Objekte wird aus den unterschiedlichen Teilansichten ein repräsentatives Modell generiert.
\end{enumerate}


Ziel der Arbeit ist es, geeignete Lösungen für die verschiedenen Ansätze zu entwickeln sowie die unterschiedlichen Rekonstruktionsmöglichkeiten zu implementieren.

Weiterhin sollen die Ergebnisse untereinander und mit bereits bestehenden Algorithmen verglichen werden.
Dieser Vergleich soll sowohl bezüglich der wichtigsten Eigenschaft der Qualität, als auch auf Basis untergeordneter Faktoren wie Geschwindigkeit der Algorithmen und Nutzungskomfort der Ansätze stattfinden.
