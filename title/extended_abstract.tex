% !TEX root = ../thesis.tex
\thispagestyle{plain}
\chapter*{Extended Abstract}
\label{ch:extended-abstract}

\begin{refsection}

\begin{center}
	\begin{tabular}{p{3cm}p{10cm}}
		Thema: & \thema \\[1ex]
		Kandidat: & \autor \\[1ex]
		Betreuer: & \prueferA \\
		 & Institut für Optische Systeme\\[1ex]
		 & \prueferB \\
		 & \firma \\[1ex]
		Abgabedatum: & \abgabedatum \\[1ex]
		Schlagworte: & \schlagworte \\[1ex]
	\end{tabular}
\end{center}


\section*{Einleitung}

Robotik spielt im industriellen Bereich eine größer werdende Rolle.
Insbesondere das sogenannte Bin-Picking, also das autonome Greifen unsortierter Teile gewinnt zunehmend an Relevanz.
Zur Erkennung der Objekte werden dabei in vielen Fällen CAD-Modelle benötigt, welche jedoch nicht immer vorliegen.
Die Rekonstruktion aus aufgenommenen 3D-Daten bietet daher eine einfache Möglichkeit, diese zu erhalten.
Es wurden verschiedene Ansätze dafür analysiert, verglichen und bewertet.


\section*{Methodik}




%TODO

\printbibliography[heading=subbibliography]

\end{refsection}
\newpage
