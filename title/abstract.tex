% !TEX root = ../thesis.tex
\thispagestyle{plain}
\chapter*{Abstract}
\label{ch:abstract}


\begin{center}
	\begin{tabular}{p{3cm}p{10cm}}
		Thema: & \thema \\[1ex]
		% & \\
		Kandidat: & \autor \\[1ex]
		% & \\
		Betreuer: & \prueferA \\%[.5ex]
		 & Institut für Optische Systeme\\[1ex]
		 & \prueferB \\%[.5ex]
		 & \firma \\[1ex]
		% & \\
		Abgabedatum: & \abgabedatum \\[1ex]
		% & \\
		Schlagworte: & \schlagworte \\
		% & \\
	\end{tabular}
\end{center}


Es wurden verschiedene Ansätze zur Rekonstruktion von Meshes aus 3D-Punktwolken analysiert, miteinander verglichen und bewertet.
Dazu wurde eine Pipeline entwickelt, welche aus Segmentierung, Registrierung und Triangulation der Punktwolke besteht.

Im ersten Ansatz wurde ein einzelnes Teil mithilfe einer beweglichen Kamera am Roboterarm aus mehreren Posen eingescannt und so ein Gesamtbild erstellt.
Bei der zweiten Methode wurde zusätzlich eine Registrierung durchgeführt, sodass eine fest montierte Kamera verwendet werden kann.
Im dritten Ansatz fand zunächst eine Segmentierung statt, welche sortenreine Teile in einzelne Cluster teilt.
Nach der Registrierung dieser Segmente wurde eine Triangulation durchgeführt.

Es hat sich gezeigt, dass die Qualität des generierten Meshs entscheidend von der Segmentierung abhängig ist.
Die Rekonstruktion durch Rotation eines einzelnen Objekts lieferte das beste Ergebnis.


\newpage
