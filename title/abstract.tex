% !TEX root = ../thesis.tex
\thispagestyle{plain}
\chapter*{Abstract}
\label{ch:abstract}


\begin{center}
	\begin{tabular}{p{3.2cm}p{9.6cm}}
		Thema: & \thema \\[1ex]
		% & \\
		Bachelorkandidat: & \autor \\[1ex]
		% & \\
		Betreuer: & \prueferA \\%[.5ex]
		 & Institut für Optische Systeme\\[1ex]
		 & \prueferB \\%[.5ex]
		 & \firma \\[1ex]
		% & \\
		Abgabedatum: & \abgabedatum \\[1ex]
		% & \\
		Schlagworte: & \schlagworte \\
		% & \\
	\end{tabular}
\end{center}


Describe the objective and results of this thesis in a few words.
Typically one page.

In dieser Arbeit werden verschiedene Ansätze zur Rekonstruktion von Meshes aus 3D-Punktwolken analysiert, miteinander verglichen und bewertet.
Die erste Methode ist YAK, eine Variante von Kinect Fusion, bei der ein Truncated Signed Distance Field zum Einsatz kommt.
Der zweite Ansatz ist, ein einzelnes Teil mithilfe einer beweglichen Kamera am Roboterarm aus mehreren Posen einzuscannen und so ein Gesamtbild zu erhalten.
Im dritten Ansatz wird eine 

\newpage
